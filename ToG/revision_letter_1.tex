%%%%%%%%%%%%%%%%%%%%%%%%%%%%%%%%%%%%%%%%%
% Thin Formal Letter
% LaTeX Template
% Version 2.0 (7/2/17)
%
% This template has been downloaded from:
% http://www.LaTeXTemplates.com
%
% Author:
% Vel (vel@LaTeXTemplates.com)
%
% Originally based on an example on WikiBooks 
% (http://en.wikibooks.org/wiki/LaTeX/Letters) but rewritten as of v2.0
%
% License:
% CC BY-NC-SA 3.0 (http://creativecommons.org/licenses/by-nc-sa/3.0/)
%
%%%%%%%%%%%%%%%%%%%%%%%%%%%%%%%%%%%%%%%%%

%----------------------------------------------------------------------------------------
%	DOCUMENT CONFIGURATIONS
%----------------------------------------------------------------------------------------

\documentclass[10pt]{letter} % 10pt font size default, 11pt and 12pt are also possible

\usepackage{geometry} % Required for adjusting page dimensions

%\longindentation=0pt % Un-commenting this line will push the closing "Sincerely," to the left of the page

\geometry{
	paper=a4paper, % Change to letterpaper for US letter
	top=3cm, % Top margin
	bottom=1.5cm, % Bottom margin
	left=4.5cm, % Left margin
	right=4.5cm, % Right margin
	%showframe, % Uncomment to show how the type block is set on the page
}

\usepackage[T1]{fontenc} % Output font encoding for international characters
\usepackage[utf8]{inputenc} % Required for inputting international characters

\usepackage{stix} % Use the Stix font by default

\usepackage{microtype} % Improve justification

%----------------------------------------------------------------------------------------
%	YOUR NAME & ADDRESS SECTION
%----------------------------------------------------------------------------------------

\signature{Mohammed SALEM} % Your name for the signature at the bottom

\address{University of Mascara \\ Algeria\\ salem@univ-mascara.dz}

%----------------------------------------------------------------------------------------

\begin{document}

%----------------------------------------------------------------------------------------
%	ADDRESSEE SECTION
%----------------------------------------------------------------------------------------

\begin{letter}{Professor Julian Togelius \\ Editor in Chief, IEEE Transactions on Games} % Name/title of the addressee

%----------------------------------------------------------------------------------------
%	LETTER CONTENT SECTION
%----------------------------------------------------------------------------------------

\opening{\textbf{Dear Sir,}}


\begin{enumerate}
\item {\bf \underline{ Reviewer 1 Comments}}\\
\begin{itemize}
	\item {\bf Fitnessless:}
	\item {\bf Typos and Grammar:} Revised
	\item  {\bf Presentation Style:}
	\begin{itemize}
		\item Linebreak of the word CG-Track 1:  fixed
		\item Tables values: sorted
	\end{itemize}
		
\end{itemize}
\item {\bf \underline{ Reviewer 2 Comments}}\\

\begin{itemize}
	\item {\bf How the fitnessless method or GPS is first introduced in the introduction:}
	\item {\bf Language errors :} Revisited
	
\end{itemize}

\item {\bf \underline{ Reviewer 3 Comments}}\\

\begin{itemize}
	
	\item {\bf	There is just one track that is used for training, and the same track is used for evaluation as well (in addition to one other track)}:\\
	Authors have considered two other tracks: Wheel 2 which is a Suzuka F1 inspired track with more challenging difficulties and Street 1 track for testing purpose.
	They are multiplied the races number running the  10 races for each track.
	\item {\bf	About the  about the analysis of the noise section} However, it could have been clearer how many individuals were picked and how often they were evaluated.
	
	
		\item {\bf There were also some statements I was missing some form of evidence for:}
			\begin{itemize}
	\item	"successful racing will only need these sensor values" (2)
	\item	the approach could generalise to other types of controllers (6)
	\item	reduced diversity is good (1) -> usually, the opposite is true in my experience
	\end{itemize}

	\item {\bf 	While the paper is generally well-structured, there are several problems with the writing, sometimes making the message unclear. Examples are:}
		\begin{itemize}
		\item "A fitness is substituted by a podium"
		\item "use neuroevolution instead of evolving" -> both are evolutionary approaches?
		\item What does "map the trajectory" mean?
		\item It is not clear whether the statistics about the articles cited are about TORCS or not? And if they aren't, why are they relevant?
		\item "the higher diversification factor which it means"
\end{itemize}
	
	
	\item {\bf How the fitnessless method or GPS is first introduced in the introduction:}
	\item {\bf Minor language comments: :} 
				\begin{itemize}
			\item 	"driving a car is a problem" -> It is not clear at this point that you are talking about an optimisation problem
			\item 	"eventually some of them are decided" -> Eventually makes it sound like there is a sequence of events here?
	

			\item 	"in this environment is quite hard" -> very vague

			\item 	"on the other hand" should only be used in conjunction with "on the one hand"
			\item 	"previsely"?
			
					
					
					
		\item	"successful racing will only need these sensor values" (2)
		\item	the approach could generalise to other types of controllers (6)
		\item	reduced diversity is good (1) -> usually, the opposite is true in my experience
	\end{itemize}
	
\end{itemize}
\end{enumerate}
 

%%%%%%%%%%%%%%%%%%%%%%%%%%%%%%%%%%%%%



Thank you for your time and consideration.

I look forward to your reply.

\vspace{2\parskip} % Extra whitespace for aesthetics
\closing{Sincerely,}
\vspace{2\parskip} % Extra whitespace for aesthetics

\ps{P.S. You can find additional information attached to this letter.} % Postscript text, comment this line to remove it


%----------------------------------------------------------------------------------------

\end{letter}
 
\end{document}
